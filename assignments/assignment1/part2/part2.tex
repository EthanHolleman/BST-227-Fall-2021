\documentclass[12pt]{article}
\usepackage[utf8]{inputenc}	% Para caracteres en español
\usepackage{amsmath,amsthm,amsfonts,amssymb,amscd}
\usepackage{multirow,booktabs}
\usepackage[table]{xcolor}
\usepackage{fullpage}
\usepackage{lastpage}
\usepackage{enumitem}
\usepackage{fancyhdr}
\usepackage{mathrsfs}
\usepackage{wrapfig}
\usepackage{graphicx}
\usepackage{caption}
\usepackage{subcaption}
\usepackage{setspace}
\usepackage{calc}
\usepackage{multicol}
\usepackage{cancel}
\usepackage[T1]{fontenc} 
%\usepackage[retainorgcmds]{IEEEtrantools}
\usepackage[margin=3cm]{geometry}
\usepackage{amsmath}
\newlength{\tabcont}
\setlength{\parindent}{0.0in}
\setlength{\parskip}{0.05in}
\usepackage{empheq}
\usepackage{framed}
\usepackage[most]{tcolorbox}
\usepackage{xcolor}
\colorlet{shadecolor}{orange!15}
\parindent 0in
\parskip 12pt
\usepackage[T1]{fontenc}

 \renewcommand{\familydefault}{\sfdefault}
\geometry{margin=1in, headsep=0.25in}
\theoremstyle{definition}


\usepackage{graphicx}
\usepackage{hyperref}
\hypersetup{
	colorlinks=true,
	linkcolor=blue,
	filecolor=magenta,      
	urlcolor=cyan,
}

\begin{document}
\title{
	Assignment 1}

\author{Ethan Holleman}
\maketitle

\section{Part 2}


\begin{equation}
	\textrm{ln}P(X, C | \boldsymbol{\theta}) = \sum_{i} \sum_{j} C_{i, j} \textrm{ln}\lambda_{j} + \sum_{i}\sum_{j}\sum_{p}\sum_{k} X_{i,j,p}[C_{i,j}\textrm{ln} \psi^{(1)}_{p, k} + (1 - C_{i,j})\textrm{ln}\psi^{(0)}_{p, k}]
\end{equation}

\subsection{E step}

$P(\boldsymbol{C} | \boldsymbol{X}, \boldsymbol{\theta})$ is effectively telling us for every motif for each sequence, was this motif likely to be a transcription factor binding site given all motifs and the parameters of the model. The probability of a motif being the transcription factor binding site is ultimately a function of the nucleotides that compose that motif and the position of those nucleotides within the motif.

\begin{equation}
	P(\boldsymbol{C} | \boldsymbol{X}, \boldsymbol{\theta}) = \frac{P(\boldsymbol{X}, \boldsymbol{C} | \boldsymbol{\theta})}{P(\boldsymbol{X} | \boldsymbol{\theta})}
\end{equation}

Product individual posteriors is proportional to posterior over all latent variables.

\begin{equation}
P(\boldsymbol{C} | \boldsymbol{X}, \boldsymbol{\theta}) \propto
 \prod_{i} \prod_{j} P( X_{i, j}, C_{i} | \boldsymbol{\theta})
\end{equation}

\begin{equation}
P( X_{i, j}, C_{i} | \boldsymbol{\theta}) = \prod_{p}\prod_{k}[P(X_{i,j,p=k} | C_{i}=j, \boldsymbol{\theta})^{C_{i,j}}P(X_{i,j,p=k} | C_{i}\neq j, \boldsymbol{\theta})^{1-C_{i,j}}]^{X_{i,j,p,k}}
\end{equation}


\begin{equation}
	P(X_{i,j} | \boldsymbol{\theta}) = \prod_{j} \lambda_{j} \psi_{p, k}^{(0)} + \prod_{j} \lambda_{j} \psi_{p, k}^{(1)}
\end{equation}
 

\begin{equation}
P(\boldsymbol{C} | \boldsymbol{X}, \boldsymbol{\theta}) = \frac{\prod_{p}\prod_{k}[P(X_{i,j,p=k} | C_{i}=j, \boldsymbol{\theta})^{C_{i,j}}P(X_{i,j,p=k} | C_{i}\neq j, \boldsymbol{\theta})^{1-C_{i,j}}]^{X_{i,j,p,k}}}{\prod_{j} \lambda_{j} \psi_{p, k}^{(0)} + \prod_{j} \lambda_{j} \psi_{p, k}^{(1)}}
\end{equation}

\subsection{M step}

\begin{equation}
	\boldsymbol{E}_{q}[\textrm{log} P(\boldsymbol{X}, \boldsymbol{C} | \boldsymbol{\theta})] =  \sum_{i} \sum_{j} C_{i, j} \textrm{ln}\lambda_{j} + \sum_{i}\sum_{j}\sum_{p}\sum_{k} X_{i,j,p}[C_{i,j}\textrm{ln} \psi^{(1)}_{p, k} + (1 - C_{i,j})\textrm{ln}\psi^{(0)}_{p, k}]
\end{equation}

\begin{equation}
=  \sum_{i} \sum_{j} \boldsymbol{E}_{q} [C_{i, j} \textrm{ln}\lambda_{j}] + \sum_{i}\sum_{j}\sum_{p}\sum_{k} 	\boldsymbol{E}_{q}[X_{i,j,p}[C_{i,j}\textrm{ln} \psi^{(1)}_{p, k} + (1 - C_{i,j})\textrm{ln}\psi^{(0)}_{p, k}]]
\end{equation}

\begin{equation}
=  \sum_{i} \sum_{j} \boldsymbol{E}_{q}[C_{i, j}] \textrm{ln}\lambda_{j} + \sum_{i}\sum_{j}\sum_{p}\sum_{k} 	\boldsymbol{E}_{q}X_{i,j,p}[C_{i,j}\textrm{ln} \psi^{(1)}_{p, k} + (1 - C_{i,j})\textrm{ln}\psi^{(0)}_{p, k}]
\end{equation}

\subsubsection{Derivative with respect to $\lambda_{j}$}

Constraint equation defines bounds of $\lambda_{j}$ that we must optimize $\boldsymbol{E}_{q}[\textrm{log} P(\boldsymbol{X}, \boldsymbol{C} | \boldsymbol{\theta})]$ within. $\lambda_{j}$ is probability that a particular motif is the TFBS. There is only one TFBS per sequence so probability over all possible motifs must sum to 1.

\begin{equation}
	g(\lambda_{j}) = \sum_{j} \lambda_{j} - 1
\end{equation}


\begin{equation}
	h(\lambda_{j}) = \textrm{ELBO}(q_{t}(C), \boldsymbol{\theta}) + \phi g(\lambda_{j})
\end{equation}

\begin{equation}
	\frac{\partial{h}}{\partial{\lambda_{j}}} = \frac{\sum_{i}{E}_{q}[C_{i, j}]}{\lambda_{j}} - \phi
\end{equation}

\begin{equation}
\lambda_{j} = \frac{\sum_{i}{E}_{q}[C_{i, j}]}{\phi}
\end{equation}

Derivative with respect to $\phi$ (Lagrangian multiplier) is just the constraint equation.

\begin{equation}
	\frac{\partial{h}}{\partial{\phi}} = g(\lambda_{j})
\end{equation}

Set equal to zero and plug in value for $\lambda_{j}$.

\begin{equation}
	0 = \frac{\sum_{i}\sum_{j} {E}_{q}[C_{i, j}]}{\phi} - 1
\end{equation}

\begin{equation}
1 =  \frac{\sum_{i}\sum_{j}{E}_{q}[C_{i, j}]}{\phi}
\end{equation}

\begin{equation}
\phi = \sum_{i}  \sum_{j}{E}_{q}[C_{i, j}]
\end{equation}

Plug back into $\lambda_{j} = \frac{\sum_{i}{E}_{q}[C_{i, j}]}{\phi}$.

\begin{equation}
	\lambda_{j} = \frac{\sum_{i}{E}_{q}[C_{i, j}]}{\sum_{i}  \sum_{j}{E}_{q}[C_{i, j}]}
\end{equation}

\subsection{Derivative with respect to $\psi_{p, k}^{(0)}$}

Define constraint equation to optimize $\psi_{p, k}^{(C_{i, j})}$ within.

\begin{equation}
	g(\psi_{p, k}^{(0)}) = \sum_{k}\psi_{p, k}^{(0)} - 1 
\end{equation}

\begin{equation}
h(\psi_{p, k}^{(0)}) = \textrm{ELBO}(q_{t}(C), \boldsymbol{\theta}) + \phi g(\psi_{p, k}^{(0)})
\end{equation}

\begin{equation}
\frac{\partial{h}}{\partial{\psi_{p, k}^{(0)}}} = \frac{\sum_{i}\sum_{j}E_{q} X_{i,j,p,k} C_{i,j}}{\psi_{p, k}^{(0)}} - \phi
\end{equation}

Set derivative equal to zero and solve for $\psi_{p, k}^{(0)}$.

\begin{equation}
	\phi \psi_{p, k}^{(0)} = \sum_{i}\sum_{j}E_{q} X_{i,j,p,k} C_{i,j}
\end{equation}

\begin{equation}
\psi_{p, k}^{(0)} = \frac{\sum_{i}\sum_{j}E_{q} X_{i,j,p,k} C_{i,j}}{\phi}
\end{equation}

Take derivative of $h(\psi_{p, k}^{(0)})$ with respect to $\phi$.

\begin{equation}
\frac{\partial{h}}{\partial{\phi}} = \sum_{k}\psi_{p, k}^{(0)} - 1
\end{equation}

\begin{equation}
=\frac{\sum_{k}\sum_{i}\sum_{j}E_{q} X_{i,j,p,k} C_{i,j}}{\phi} - 1
\end{equation}

\begin{equation}
	\phi = \sum_{k}\sum_{i}\sum_{j}E_{q} X_{i,j,p,k} C_{i,j}
\end{equation}

\begin{equation}
\psi_{p, k}^{(0)} = \frac{\sum_{i}\sum_{j}E_{q} X_{i,j,p,k} C_{i,j}}{\sum_{k}\sum_{i}\sum_{j}E_{q} X_{i,j,p,k} C_{i,j}}
\end{equation}


\subsection{Derivative with respect to $\psi_{p, k}^{(0)}$}

Define constraint equation to optimize within. $\psi_{p, k}^{(C_{i, j})}$

\begin{equation}
g(\psi_{p, k}^{(1)}) = \sum_{k}\psi_{p, k}^{(1)} - 1
\end{equation}

\begin{equation}
h(\psi_{p, k}^{(1)}) = \textrm{ELBO}(q_{t}(C), \boldsymbol{\theta}) + \phi g(\psi_{p, k}^{(1)})
\end{equation}

\begin{equation}
\frac{\partial{h}}{\partial{\psi_{p, k}^{(1)}}} = \frac{\sum_{i}\sum_{j}E_{q} X_{i,j,p,k} (1 - C_{i,j})}{\psi_{p, k}^{(1)}} - \phi
\end{equation}

\begin{equation}
\psi_{p, k}^{(1)} = \frac{\sum_{i}\sum_{j}E_{q} X_{i,j,p,k} (1 - C_{i,j})}{\phi}
\end{equation}

\begin{equation}
\frac{\partial{h}}{\partial{\phi}} = \sum_{k}\psi_{p, k}^{(1)} - 1
\end{equation}

\begin{equation}
= \frac{\sum_{k} \sum_{i}\sum_{j}E_{q} X_{i,j,p,k} (1 - C_{i,j})}{\phi} - 1
\end{equation}



\begin{equation}
\psi_{p, k}^{(1)} = \frac{\sum_{i}\sum_{j}E_{q} X_{i,j,p,k} (1 - C_{i,j})}{\sum_{k} \sum_{i}\sum_{j}E_{q} X_{i,j,p,k} (1 - C_{i,j})}
\end{equation}


\end{document}
