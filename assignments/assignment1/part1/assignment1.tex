\documentclass[12pt]{article}
\usepackage[utf8]{inputenc}	% Para caracteres en español
\usepackage{amsmath,amsthm,amsfonts,amssymb,amscd}
\usepackage{multirow,booktabs}
\usepackage[table]{xcolor}
\usepackage{fullpage}
\usepackage{lastpage}
\usepackage{enumitem}
\usepackage{fancyhdr}
\usepackage{mathrsfs}
\usepackage{wrapfig}
\usepackage{graphicx}
\usepackage{caption}
\usepackage{subcaption}
\usepackage{setspace}
\usepackage{calc}
\usepackage{multicol}
\usepackage{cancel}
\usepackage[T1]{fontenc} 
%\usepackage[retainorgcmds]{IEEEtrantools}
\usepackage[margin=3cm]{geometry}
\usepackage{amsmath}
\newlength{\tabcont}
\setlength{\parindent}{0.0in}
\setlength{\parskip}{0.05in}
\usepackage{empheq}
\usepackage{framed}
\usepackage[most]{tcolorbox}
\usepackage{xcolor}
\colorlet{shadecolor}{orange!15}
\parindent 0in
\parskip 12pt
\usepackage[T1]{fontenc}

 \renewcommand{\familydefault}{\sfdefault}
\geometry{margin=1in, headsep=0.25in}
\theoremstyle{definition}


\usepackage{graphicx}
\usepackage{hyperref}
\hypersetup{
	colorlinks=true,
	linkcolor=blue,
	filecolor=magenta,      
	urlcolor=cyan,
}

\begin{document}
\title{
	Assignment 1}

\author{Ethan Holleman}
\maketitle

\section{Part 1}

\subsection{Definitions}

\begin{itemize}
	\item $i$: Index over nucleotide sequences.
	\item $P$: Length of kmers to consider in the model.
	\item $m$: Indexes over all kmers of a given nucleotide sequence.
	\item $j$: Indexes over nucleotides of a given kmer $m$.
	\item $k$: Indexes over nucleotides (A, T, G, C).
	\item $l$: Indexes over models where 0 indicates model cooresponding to transcription factor bindind motif (foreground) and 1 indicates non-binding region (background).
	\item $X_{i,m,j,k}$: indicator variable that equals 1 if nucleotide $j$ of kmer $m$ of sequence $i$ is equal to base $k$ and equals 0 otherwise.
	\item $C_{i, m, l}$: Indicator variable that will equal 1 if kmer $m$ of sequence $i$ was drawn from model $l$ and otherwise equals 0. 
	\item $\boldsymbol{X}$: The set of all $X_{i,m,j,k}$.
	\item $\boldsymbol{C}$: The set of all $C_{i, m, l}$.
	\item $P(C_{i,m,l}=1) = \lambda_{l}$ where $l$ is either 0 or 1 indicating foreground or background. 
	\item $P(X_{i,m,j,k} = 1 | C_{i,m,l}=1) = \psi^{(l)}_{j,k}$, where
	$\sum_{k}\psi^{(l)}_{j,k}=1$
	\subitem This should define two 4 by $P$ matries where each row sums to 1 and represents the probibility of observing each nucleotide at a given position $j$ of a kmer given kmer was selected from model $l$.
	\item The model parameters are denoted as $\theta$ = $\{\lambda_{l}, \psi^{(l)}_{j,k}\}$
\end{itemize}

\subsection{Complete log likelihood}

\begin{equation}
	Q(\theta | \boldsymbol{X}, \boldsymbol{C}) = \textrm{log}(P(\boldsymbol{X}, \boldsymbol{C} | \theta))
\end{equation}

\begin{equation}
	= \textrm{log} \prod_{i}\prod_{m}P(X_{i,m}, C_{i, m} | \theta)
\end{equation}

\begin{equation}
= \textrm{log} \prod_{i}\prod_{m}(\prod_{j}\prod_{k}\prod_{l}[P(C_{i,m,l}=1|\theta)P(X_{i,m,j,k}=1|C_{i,m,l}=1, \theta)]^{X_{i, m, j, k}C_{i, m, l}})
\end{equation}

\begin{equation}
= \textrm{log} \prod_{i}\prod_{m}(\prod_{j}\prod_{k}\prod_{l}[\lambda_{l}\psi_{j, k}^{(l)}])
\end{equation}

\begin{equation}
= \sum_{i}\sum_{m}\sum_{j}\sum_{k}\sum_{l}X_{i,m,j,k}C_{i, m, l}\textrm{log}[\lambda_{l}\psi_{j, k}^{(l)}]
\end{equation}

\begin{equation}
= \sum_{i}\sum_{m}\sum_{j}\sum_{k}\sum_{l}X_{i,m,j,k}C_{i, m, l} \textrm{log}\lambda_{l} + \sum_{i}\sum_{m}\sum_{j}\sum_{k}\sum_{l}X_{i,m,j,k}C_{i, m, l}\textrm{log}\psi_{j, k}^{(l)}
\end{equation}







\end{document}

