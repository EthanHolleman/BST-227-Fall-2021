\documentclass[12pt]{article}
\usepackage[utf8]{inputenc}	% Para caracteres en español
\usepackage{amsmath,amsthm,amsfonts,amssymb,amscd}
\usepackage{multirow,booktabs}
\usepackage[table]{xcolor}
\usepackage{fullpage}
\usepackage{lastpage}
\usepackage{enumitem}
\usepackage{fancyhdr}
\usepackage{mathrsfs}
\usepackage{wrapfig}
\usepackage{graphicx}
\usepackage{caption}
\usepackage{subcaption}
\usepackage{setspace}
\usepackage{calc}
\usepackage{multicol}
\usepackage{cancel}
\usepackage[T1]{fontenc} 
%\usepackage[retainorgcmds]{IEEEtrantools}
\usepackage[margin=3cm]{geometry}
\usepackage{amsmath}
\newlength{\tabcont}
\setlength{\parindent}{0.0in}
\setlength{\parskip}{0.05in}
\usepackage{empheq}
\usepackage{framed}
\usepackage[most]{tcolorbox}
\usepackage{xcolor}
\colorlet{shadecolor}{orange!15}
\parindent 0in
\parskip 12pt
\usepackage[T1]{fontenc}

 \renewcommand{\familydefault}{\sfdefault}
\geometry{margin=1in, headsep=0.25in}
\theoremstyle{definition}


\usepackage{graphicx}
\usepackage{hyperref}
\hypersetup{
	colorlinks=true,
	linkcolor=blue,
	filecolor=magenta,      
	urlcolor=cyan,
}

\begin{document}
\title{
	Assignment 1}

\author{Ethan Holleman}
\maketitle

\section{Part 1}

\subsection{Definitions}

\begin{itemize}
	\item $i$: Index over nucleotide (input) sequences.
	\item $P$: Length of motif to consider in the model.
	\item $j$: Indexes over start positions of all possible motifs in a given sequence $i$ of length $L$ assuming all sequences have length $L$. $0 \leq j \leq L - P+1$
	\item $m$: Indexes of each nucleotide of motif $j$ of sequence $i$. $0 \leq m \leq P$.
	\item $l$: Indexes over models where 1 indicates model corresponding to transcription factor binding motif (foreground) and 0 indicates non-binding region (background).
	\item $k$: Indexes over nucleotides (A, T, G, C).
	\item $X_{i,j,m,k}$: Indicator variable in which $X_{i,j,m,k}=1$ if base $m$ of the motif beginning at position $j$ of sequence $i$ is equal to nucleotide $k$ and is 0 otherwise.
	\item $C_{i}$: Vector of motif start positions for each sequence $i$. If $C_{i} = j$ then then motif $j$ is the transcription factor binding site.
	\subitem We use the indicator variable $C_{i,j}=1$ to represent when $C_{i} = j$.
	\item $P(C_{i,j} = l) = \lambda_{j}$ and $0 \leq \lambda_{j} \leq 1$.
	\item $P(X_{i,j,m,k}=l | C_{i,j}=l) = \psi_{k, m}^{(l)}$ where $\sum_{k}\psi^{(l)}_{k ,m}=1$
	\item $\boldsymbol{X}$: The set of all $X_{i,j,m,k}$.
	\item $\boldsymbol{C}$: The set of all $C_{i}$.
	\item The set of all model parameters is denoted as $\boldsymbol{\theta} = \{\lambda_{j}, \psi^{(l)}_{k ,m}\}$
\end{itemize}

\subsection{Complete log likelihood}

\begin{equation}
	Q(\theta | \boldsymbol{X}, \boldsymbol{C}) = \textrm{log}(P(\boldsymbol{X}, \boldsymbol{C} | \theta))
\end{equation}

\begin{equation}
	= \textrm{log} \prod_{i}\prod_{j}P(X_{i,j}, C_{i,j} | \theta)
\end{equation}

\begin{equation}
	= \textrm{log}\prod_{i}\prod_{j} (\prod_{m}\prod_{k}\prod_{l}[P(X_{i,j,m,k}=1 | C_{i, j}=l, \theta)P(C_{i, j} = l | \theta)]^{X_{i,j,m,k} C_{i, j}})
\end{equation}

\begin{equation}
= \textrm{log} \prod_{i} \prod_{j} (\prod_{m} \prod_{k}\prod_{l}[\lambda_{j} \psi_{k, m}^{(l)}])^{X_{i,j,m,k} C_{i, j}}
\end{equation}

\begin{equation}
	=\sum_{i} \sum_{j} \sum_{m} \sum_{k} \sum_{l} X_{i,j,m,k} C_{i,j} \textrm{log}[\psi_{k, m}^{(l)}]
\end{equation}

\begin{equation}
=\sum_{i} \sum_{j} \sum_{m} \sum_{k} \sum_{l} X_{i,j,m,k} C_{i,j} \textrm{log}[\psi_{k, m}^{(l)}]
\end{equation}

\begin{equation}
=\sum_{i} \sum_{j} \sum_{m} \sum_{k} \sum_{l} X_{i,j,m,k} C_{i,j} \textrm{log} \lambda_{j} + \sum_{i} \sum_{j} \sum_{m} \sum_{k} \sum_{l} X_{i,j,m,k} C_{i,j} \textrm{log} \psi_{k, m}^{(l)}
\end{equation}























\end{document}
